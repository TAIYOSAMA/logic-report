%プリアンブル
%ドキュメントクラス
\documentclass{jlreq}    %このオプションで.pngや.jpgが使えるようになる

%箇条書きの拡張
\usepackage{enumerate}
\usepackage{enumitem}

%数学系
\usepackage{amsmath}
\usepackage{mathtools}
\usepackage{empheq}
\usepackage{amssymb}

%tableを指定した場所に確実に描画する
\usepackage{here}
\usepackage{booktabs}
\usepackage{array}

%画像を挿入
\usepackage{graphicx}

%URLに使われた記号でコンパイルに詰まる事態の回避
\usepackage{url}


% 定理環境の初期化
\usepackage{amsthm}

\makeatletter
\def\thm@space@setup{%
  \thm@preskip=\medskipamount
  \thm@postskip=\medskipamount
}
\makeatother

% \theoremstyle{plain}
% \newtheorem{thm}{Theorem}[subsection]

% \theoremstyle{definition}
% \newtheorem{dfn}[thm]{Definition}

% \theoremstyle{definition}
% \newtheorem{ex}[thm]{Example}

% \theoremstyle{remark}
% \newtheorem{rem}[thm]{Remark}

% \theoremstyle{definition}
% \newtheorem{ax}[thm]{Axiom}

\theoremstyle{plain}
\newtheorem{thm}{定理}[subsection]

\theoremstyle{definition}
\newtheorem{dfn}[thm]{定義}

\theoremstyle{definition}
\newtheorem{ex}[thm]{例}

\theoremstyle{remark}
\newtheorem{rem}[thm]{注意}

\theoremstyle{definition}
\newtheorem{ax}[thm]{公理}

\theoremstyle{definition}
\newtheorem{prb}{小問}


\newcounter{step}[subsection]
\renewcommand{\thestep}{\arabic{step}}

\newcommand{\STEP}[1]{%
  \refstepcounter{step}%
  \par\medskip\medskip
  \noindent\textbf{STEP.\thestep\ #1}
}


%本体
\begin{document}

    \part*{論理学レポート}

\begin{description}
    \item[学籍番号] C5TB2006
    \item[氏名] 安倍 大陽
\end{description}

\pagebreak           % 表紙
    \part{授業内容のまとめ}

\section*{論理学とは}

論理学とは理屈を述べるための理屈であり、前提と結論に関する学問のことである。
また、実生活の役に立つことは無いに等しく、学ぶという事象に満足する学問であるとも言えるかもしれない。
効率化が騒がれる現代で論理学を学ぶことは非効率・非現代的の象徴であり愚行とも捉えられかねないが、
自分なりの学ぶ意味を再構築するという点において重要な意味を持つ。かもしれない。

実生活の役に立たない論理学であるが、学問においてその多岐性は目を見張る物がある。
特に情報領域においてはコンピュータの発展に大きく寄与しており、現代技術を支えている。
現代という時代を支える一方で、学問としてはアンチ現代的であるがゆえに、矛盾的葛藤を内包した学問とも言える。かもしれない。
また、数学、哲学の発展にも大きく発展したという歴史がある。論理学が確立した理論体系は様々な学問に幅広く影響を与えている。

論理学の特徴としてそのとっつきやすさが挙げられる。
勉強を始めてから最新分野へ至るまで網羅すべき知識が他学問に比べても限られており、
勉強しようと意気込んでからパイオニアになるまであまり時間を要しないという学問上大きなメリットも存在している。

論理学は立場によって捉え方が変わるという特徴もある。
コンピューターサイエンス、哲学専攻の人たちにとっての論理学は、論理学者にとっての論理学とは別物である。
今回の講義ではあくまで論理学者にとっての論理学、すなわち前提と結論に関する学問としての論理学を学修した。

\medskip
\section*{学修の手順}

本講義は全ての論理に対して、以下の手順で進められた。

\begin{enumerate}[label=]
  \item STEP.1 \hspace{.5cm} 形式言語の導入
  \item STEP.2 \hspace{.5cm} 意味論
  \item STEP.3 \hspace{.5cm} 証明論
  \item STEP.4 \hspace{.5cm} 意味論と証明論の関係
\end{enumerate}

数学科においては意味論より先に証明論を学ぶ場合が多いが、本講義においては意味論から証明論という順で
論理学について学修していった。

\medskip \pagebreak
\subsection{古典命題論理}

まず初めに、論理学が扱う論理とは何なのか、例に従って簡単に確認する。

\medskip

\begin{ex}\mbox{}
  \vspace{-0.5\baselineskip}
  \begin{itemize}
    \item「$1 + 1 = 2$」, 「$1 + 1 = 2$」ならば『$1 + 1 + 1 = 3$』\\
          ゆえに、『$1 + 1 + 1 = 3$』
    \item「仙台は日本の首都である」, 「仙台は日本の首都である」ならば『東北地方に日本の首都はある』\\
          ゆえに、『東北地方に日本の首都はある』
  \end{itemize}
\end{ex}

\medskip

1つ目の例は確からしく見える。一方で、2つ目の例は一見すると違和感がある。仙台は日本の首都ではないからだ。
しかし、論理の構造に着目すると、2つの例はどちらも同じ形をしていることがわかる。
論理学とは、こうした論理の構造に焦点を当てた学問なのである。
以下に授業で扱ったもう1つの例を提示しておく。

\medskip

\begin{ex}\mbox{}
  \vspace{-0.5\baselineskip}
  \begin{itemize}
    \item「nは偶数である」ならば『$n^2+3n+1$ は奇数である。』\\
          ゆえに、『$n^2+3n+1$は奇数でない』ならば「nは偶数ではない。」
    \item「大森は仙台に住んでいる」ならば『大森は日本に住んでいる』\\
          ゆえに、『大森は日本に住んでいない』ならば「大森は仙台に住んでいない」
  \end{itemize}
\end{ex}

この例においても、どちらも対偶の形であるという点で共通している。
論理学ではこうした論理の構造を形式言語を用いて表現していく。


\subsection{様相論理}

様相論理を学修した。
\section{述語論理}

\subsection{古典述語論理}

\STEP{形式言語の導入}

\begin{dfn}
  古典述語論理の言語$\mathcal{L}_{PL}$(Predicate Logic)は次のものからなる。
  \begin{itemize}
    \item 個体変項: $x , y , z , \dots$
    \item 個体定項: $a, b, c, \dots$
    \item 述語: $P, Q, R, \dots$
    \item 論理結合子: $\lnot, \land, \lor, \to$
    \item 量化子: $\forall, \exists$
    \item カッコ: $(, )$
  \end{itemize}
\end{dfn}

\begin{dfn}
  言語$\mathcal{L}_{PL}$が与えられたとき、論理式を以下のように定める。
  \begin{itemize}
    \item $t_1, \dots, t_n$が項であり、$P$が$n$項連続のとき、$P_{t_1 \dots t_n}$は論理式である。(とくに原子式と呼ばれる)
    \item 2つの論理式$A, B$があるとき、$A$, $(A \land B)$, $(A \lor B)$, $(A \to B)$は論理式である。
    \item 論理式$A$と個体変項があるとき、$\forall_{x}A, \exists_{x}A$は論理式である。
  \end{itemize}
  上の3つの条件によってのみ論理式は決まる。
\end{dfn}

\STEP{意味論}

\begin{dfn}
  言語$\mathcal{L}_{PL}$が与えられているとき、個体変項$x$の出現が束縛されているとは、
  \begin{equation}
    \forall_x \dots x \dots
  \end{equation}
  あるいは、
  \begin{equation}
    \exists_x \dots x \dots
  \end{equation}
  のように出現することを言う。
  個体変項$x$の出現が束縛されていないとき、$x$は自由であるという。
  すべての個体変項の出現が束縛されている論理式を閉じた論理式という。
  また、論理式$A$において、自由に出現する個体変項$x$をすべて項$t$で置き換えて得られる論理式を$A_x(t)$とかく。
\end{dfn}

\begin{dfn}
  言語$\mathcal{L}_{PL}$に対する解釈とは、組$\mathfrak{I}=<D, I>$である。このとき、
  \begin{itemize}
    \item $D$: 空でない集合
    \item $I$: 関数であって、個体定項$C$に対して$I(C)$は$D$に含まれる元であり、$n$項述語$P$に対して、$I(P)$は$D^n$の部分集合
  \end{itemize}
  をそれぞれ対応させるものである。
\end{dfn}

\begin{dfn}
  言語$\mathcal{L}_{PL}$に対する$\mathfrak{I} = <D, I>$が与えられたとき、$\mathcal{L}_{PL}(\mathfrak{I})$における閉じた論理式全体から$\{T, F\}$への関数$v$を以下のように定める。
  \begin{itemize}
    \item $v({P_0}_{a_1 \dots a_n}) = T \iff <I(a_1), \dots, I(a_n)> \in I(P)$
    \item $v({P_0}_{a_1 \dots a_n}) = F \iff <I(a_1), \dots, I(a_n)> \notin I(P)$
  \end{itemize}
\end{dfn}

\begin{align}
  v(\lnot A) = T &\iff v(A) = F\\
  v(\lnot A) = F &\iff v(A) = T\\
  v(A \land B) = T &\iff v(A) = T \ かつ \ v(B) = T\\
  v(A \land B) = F &\iff v(A) = F \ または \ v(B) = F\\
  v(A \lor B) = F &\iff v(A) = T \ または \ v(B) = T\\
  v(A \lor B) = F &\iff v(A) = F \ かつ \ v(B) = F\\
  v(A \to B) = T &\iff v(A) = F \ または \ v(B) = T\\
  v(A \to B) = F &\iff v(A) = T \ かつ \ v(B) = F\\
  v(\forall_xA) = T &\iff すべてのDの元dに対して、v(A_x(k_d)) = T\\
  v(\forall_x A) = F &\iff ある D の元 d に対して、v(A_x(k_d)) = F\\
  v(\exists_x A) = T &\iff ある D の元 d に対して、v(A_x(k_d)) = T\\
  v(\exists_x A) = F &\iff すべての D の元 d に対して、v(A_x(k_d)) = F
\end{align}

\begin{dfn}
  $A_1, \dots, A_n, B$を閉じた論理式とする。このとき、
  $A_1, \dots, A_n \models_{PL} B \iff v(A_1) = T, \dots, v(A_n) = T かつ v(B) = F$となるような解釈$\mathfrak{I}$がない
  
  ここで、$\models_{PL}$を古典述語論理の意味論的帰結関係という。
\end{dfn}

\pagebreak
   % 授業内容のまとめ
    \section{講義を受けての感想}

様相論理はよくわからなかった。


\pagebreak  % 感想
    \part{演習問題}

\begin{prb}
  $\models((A \to B) \to A) \to A$
  \begin{equation}
    v_0(((A \to B) \to A) \to A) = F \label{eq:1-1}
  \end{equation}
  となる解釈$I_0$があるとする。このとき\eqref{eq:1-1}と$\to$の偽条件より、
  \begin{align}
    v_0((A \to B) \to A) = T \label{eq:1-2} \\
    v_0(A) = F \label{eq:1-3}
  \end{align}
  が同時に成り立つ。また、\eqref{eq:1-2}と$\to$の真理条件から、
  \begin{align}
    v_0(A \to B) = F \label{eq:1-4} \\
    v_0(A) = T \label{eq:1-5}
  \end{align}
  の少なくとも一方は成り立つ。このとき、\eqref{eq:1-4}と$\to$の偽条件より、
  \begin{align}
    v_0(A) = T \label{eq:1-6} \\
    v_0(B) = F \label{eq:1-7}
  \end{align}
  が同時に成り立つ。しかし、\eqref{eq:1-4}のときは\eqref{eq:1-3}と\eqref{eq:1-6}より矛盾し、\eqref{eq:1-5}のときも\eqref{eq:1-3}と矛盾する。\\
  したがって、解釈$I_0$は存在しない。$\qed$
\end{prb}

\medskip
\medskip

\begin{prb}
  $\models((A \to B) \lor (B \to C))$
  \begin{equation}
    v_0((A \to B) \lor (B \to C)) = F \label{eq:2-1}
  \end{equation}
  となる解釈$I_0$があるとする。\eqref{eq:2-1}と$\lor$の偽条件より、
  \begin{align}
    v_0(A \to B) = F \label{eq:2-2} \\
    v_0(B \to C) = F \label{eq:2-3}
  \end{align}
  が同時に成り立つ。このとき、\eqref{eq:2-2}と$\to$の偽条件から、
  \begin{align}
    v_0(A) = T \label{eq:2-4} \\
    v_0(B) = F \label{eq:2-5}
  \end{align}
  が同時に成り立つ。また、\eqref{eq:2-3}と$\to$の偽条件から、
  \begin{align}
    v_0(B) = T \label{eq:2-6} \\
    v_0(C) = F \label{eq:2-7}
  \end{align}
  が同時に成り立つ。しかし、\eqref{eq:2-5}と\eqref{eq:2-6}は矛盾する。\\
  したがって解釈$I_0$は存在しない。$\qed$
\end{prb}

\medskip
\medskip

\begin{prb}
  $\models \lnot (A \to B) \to A$
  \begin{equation}
    v_0(\lnot (A \to B) \to A) = F \label{eq:3-1}
  \end{equation}
  となる解釈$I_0$があるとする。\eqref{eq:3-1}と$\to$の偽条件より、
  \begin{align}
    v_0(\lnot (A \to B)) = T \label{eq:3-2} \\
    v_0(A) = F \label{eq:3-3}
  \end{align}
  が同時に成り立つ。\eqref{eq:3-2}と$\lnot$の真理条件から、
  \begin{equation}
    v_0(A \to B) = F \label{eq:3-4}
  \end{equation}
  \eqref{eq:3-4}と$\to$の偽条件より、
  \begin{align}
    v_0(A) = T \label{eq:3-5} \\
    v_0(B) = F \label{eq:3-6}
  \end{align}
  が同時に成り立つ。しかし、\eqref{eq:3-3}と\eqref{eq:3-5}は矛盾する。\\
  したがって、解釈$I_0$は存在しない。$\qed$
\end{prb}

\medskip
\medskip

\begin{prb}
  $\models((A \to B) \to B) \to A$
  \begin{equation}
    v_0(((A \to B) \to B) \to A) = F \label{eq:4-1}
  \end{equation}
  となる解釈$I_0$があるとする。このとき\eqref{eq:4-1}と$\to$の偽条件より、
  \begin{align}
    v_0((A \to B) \to B) = T \label{eq:4-2} \\
    v_0(A) = F \label{eq:4-3}
  \end{align}
  が同時に成り立つ。また、\eqref{eq:4-2}と$\to$の真理条件から、
  \begin{align}
    v_0(A \to B) = F \label{eq:4-4} \\
    v_0(B) = T \label{eq:4-5}
  \end{align}
  の少なくとも一方は成り立つ。このとき、\eqref{eq:4-4}が成り立つと仮定すると$\to$の偽条件より、
  \begin{align}
    v_0(A) = T \label{eq:4-6} \\
    v_0(B) = F \label{eq:4-7}
  \end{align}
  が同時に成り立つ。しかしこれは、\eqref{eq:4-3}と\eqref{eq:4-6}で矛盾が生じる。\\
  一方、\eqref{eq:4-4}が成り立たないと仮定すると$v_0(A \to B) = T$であり、これと$\to$の真理条件から、
  \begin{align}
    v_0(A) = F \label{eq:4-8} \\
    v_0(B) = T \label{eq:4-9}
  \end{align}
  の少なくとも一方が成り立つ。このとき、$v_0(A)=F$、$v_0(B)=T$であれば矛盾は生じず解釈$I_0$を定めることができる。
  このとき反例は$v_0(A)=F$、$v_0(B)=T$ $\qed$
\end{prb}

\medskip
\medskip

\begin{prb}
  $A \to B \models \lnot A \to \lnot B$
  \begin{align}
    v_0(A \to B) = T \label{eq:5-1} \\
    v_0(\lnot A \to \lnot B) = F \label{eq:5-2}
  \end{align}
  が同時に成り立つ解釈$I_0$があるとする。このとき、\eqref{eq:5-2}と$\to$の偽条件より、
  \begin{align}
    v_0(\lnot A) = T \label{eq:5-5} \\
    v_0(\lnot B) = F \label{eq:5-6}
  \end{align}
  が同時に成り立つ。また\eqref{eq:5-5}、\eqref{eq:5-6}それぞれについて$\lnot$の真理、偽条件から、
  \begin{align}
    v_0(A) = F \label{eq:5-7} \\
    v_0(B) = T \label{eq:5-8}
  \end{align}
  が同時に成り立つ。実際、$v_0(A)=F$、$v_0(B)=T$と定めれば\eqref{eq:5-1}においても矛盾は生じず解釈$I_0$を定めることができる。
  したがって、反例は$v_0(A)=F$、$v_0(B)=T$ $\qed$
\end{prb}   % 小問
    
\end{document}
