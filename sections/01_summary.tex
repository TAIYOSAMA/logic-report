\subsection{授業内容まとめ}

\subsubsection{論理学とは}

論理学とは理屈を述べるための理屈であり、前提と結論に関する学問のことである。
また、実生活の役に立つことは無いに等しく、学ぶという事象に満足する学問であるとも言えるかもしれない。
効率化が騒がれる現代で論理学を学ぶことは非効率・非現代的の象徴であり愚行とも捉えられかねないが、
自分なりの学ぶ意味を再構築するという点において重要な意味を持つ。かもしれない。

実生活の役に立たない論理学であるが、学問においてその多岐性は目を見張る物がある。
特に情報領域においてはコンピュータの発展に大きく寄与しており、現代技術を支えている。
現代という時代を支える一方で、学問としてはアンチ現代的であるがゆえに、矛盾的葛藤を内包した学問とも言える。かもしれない。
また、数学、哲学の発展にも大きく発展したという歴史がある。論理学が確立した理論体系は様々な学問に幅広く影響を与えている。

論理学の特徴としてそのとっつきやすさが挙げられる。
勉強を始めてから最新分野へ至るまで網羅すべき知識が他学問に比べても限られており、
勉強しようと意気込んでからパイオニアになるまであまり時間を要しないという学問上大きなメリットも存在している。

\subsubsection{古典命題論理}
古典命題論理を学習した。

\pagebreak
