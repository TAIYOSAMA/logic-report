\part{講義を受けての感想}

論理学の講義は非常に面白かった。
一見すると当たり前に思えることを理屈立てて考え直す姿勢は特に印象に残っている。
先生は論理学は学んだところで意味はないとおっしゃっていたが、当たり前を疑う姿勢を授業を通して感じ取れたので、
個人的には意味のある講義だと感じた。授業内容については、各論理に対して同じステップを適用していたので、
難しくはあるもののあまり抵抗なく進められた。しかし、様相論理は可能世界の扱い方にまだ慣れないと感じたので、もう少し自分で勉強してみたいと思う。
また、授業内容とはあまり関係はないが、冒頭の雑談タイム然り先生の人柄が個人的に興味深かった。
生業としている学問に対して意味がないと言ってみたり、雑談すべきか迷いながら雑談したりしている姿から、
一旦立ち止まって当たり前を疑ってかかる論理学者の気概を感じられワクワクした。
悪あがきはしてみるもんだという発言には正直感動した。私はたびたび講義を休んでしまっていたため、
単位を取れるのか不安、仮に取れたとしても申し訳ない、と感じていた。
しかし、この言葉を聞いて、できるだけ悪あがきしてみようと思えた。実際今こうしてレポートをなんとなく
そこそこ真面目に後ろめたくなく書けている。以前の私ならば、おこがましくはあるが、授業を休んだ身分で少しでもいい成績を取ろうとするのは申し訳ないと思っていた。
だが頑張ってみたいという葛藤があった。授業中何気なく放った一言かもしれないが、私にとっては結構印象に残った言葉であり、
暫くの間は足掻いてみてもいいと思えた。そんな性根で質問しに行った際も、暖かく接してくださって、悪あがきも悪くないと思えてしまった。
そんなこんなで諦めずにしっかり向き合う姿勢を確立していけたらと思う。

論理学を学ぶだけでなく、学びの姿勢に対し自分に問い、現時点での答えをある程度しっかりとしたものにできたので、この講義を受けてよかったと心から思っている。



\pagebreak