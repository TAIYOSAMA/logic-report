\subsection{古典命題論理}

まず初めに、論理学が扱う論理とは何なのか、例に従って簡単に確認する。

\medskip

\begin{ex}\mbox{}
  \vspace{-0.5\baselineskip}
  \begin{itemize}
    \item「$1 + 1 = 2$」, 「$1 + 1 = 2$」ならば『$1 + 1 + 1 = 3$』\\
          ゆえに、『$1 + 1 + 1 = 3$』
    \item「仙台は日本の首都である」, 「仙台は日本の首都である」ならば『東北地方に日本の首都はある』\\
          ゆえに、『東北地方に日本の首都はある』
  \end{itemize}
\end{ex}

\medskip

1つ目の例は確からしく見える。一方で、2つ目の例は一見すると違和感がある。仙台は日本の首都ではないからだ。
しかし、論理の構造に着目すると、2つの例はどちらも同じ形をしていることがわかる。
論理学とは、こうした論理の構造に焦点を当てた学問なのである。
以下に授業で扱ったもう1つの例を提示しておく。

\medskip

\begin{ex}\mbox{}
  \vspace{-0.5\baselineskip}
  \begin{itemize}
    \item「nは偶数である」ならば『$n^2+3n+1$ は奇数である。』\\
          ゆえに、『$n^2+3n+1$は奇数でない』ならば「nは偶数ではない。」
    \item「大森は仙台に住んでいる」ならば『大森は日本に住んでいる』\\
          ゆえに、『大森は日本に住んでいない』ならば「大森は仙台に住んでいない」
  \end{itemize}
\end{ex}

この例においても、どちらも対偶の形であるという点で共通している。
論理学ではこうした論理の構造を形式言語を用いて表現していく。

