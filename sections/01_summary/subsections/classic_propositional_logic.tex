\section{古典命題論理}

\subsection{論理学が扱う論理}

まず初めに、論理学が扱う論理とは何なのか、例に従って簡単に確認する。

\begin{ex}(推論)
  \begin{itemize}
    \item「$1 + 1 = 2$」, 「$1 + 1 = 2$」ならば『$1 + 1 + 1 = 3$』\\
          ゆえに、『$1 + 1 + 1 = 3$』
    \item「仙台は日本の首都である」, 「仙台は日本の首都である」ならば『東北地方に日本の首都はある』\\
          ゆえに、『東北地方に日本の首都はある』
  \end{itemize}
\end{ex}

1つ目の例は確からしく見える。一方で、2つ目の例は一見すると違和感がある。仙台は日本の首都ではないからだ。
しかし、論理の構造に着目すると、2つの例はどちらも同じ形をしていることがわかる。
論理学とは、こうした論理の構造に焦点を当てた学問なのである。
以下に授業で扱ったもう1つの例を提示しておく。

\begin{ex}(推論)
  \begin{itemize}
    \item「nは偶数である」ならば『$n^2+3n+1$ は奇数である。』\\
          ゆえに、『$n^2+3n+1$は奇数でない』ならば「nは偶数ではない。」
    \item「大森は仙台に住んでいる」ならば『大森は日本に住んでいる』\\
          ゆえに、『大森は日本に住んでいない』ならば「大森は仙台に住んでいない」
  \end{itemize}
\end{ex}

この例においても、どちらも対偶の形であるという点で共通している。
論理学ではこうした論理の構造を形式言語を用いて表現していく。



\pagebreak

\subsection{古典命題論理}
論理学の中でももっとも基本的な古典命題論理(Classic Propositional Logic)についてまとめる。

% --------------------------------------------形式言語の導入--------------------------------------------------

\STEP{形式言語の導入}

\begin{dfn}
  古典命題論理の言語$\mathcal{L}_{CPL}$(Classic Propositional Logic)は以下のものからなる。
  \begin{itemize}
    \item 命題変数: $p, q, r, \dots$
    \item 論理結合子: $\lnot, \land, \lor, \to $
    \item カッコ: $\lparen, \rparen$
  \end{itemize}
  ここで、論理演算子はそれぞれ、$\lnot$は否定、$\land$はかつ、$\lor$はまたは、$\to$はならばをそれぞれ表す。
\end{dfn}

\begin{dfn}
  言語$\mathcal{L}_{CPL}$が与えられているとき、論理式を以下のように定める。
  \begin{itemize}
    \item 命題変数は論理式である
    \item 2つの論理式$A, B$が与えられているとき、$\lnot A, (A \land B), (A \lor B), (A \to B)$も論理式である。
    \item 上の2つの条件によってのみ論理式は決まる。
  \end{itemize}
\end{dfn}

これらによって言語$\mathcal{L}_{CPL}$は定められる。以下は定義に沿った例である。

\begin{ex}(論理式)
  \begin{itemize}
    \item $P$は論理式である。
    \item $\lnot P, (\lnot P \to \lnot \lnot P), \lnot(P \to P)$は論理式である。
    \item $P \lnot \to \land$や$\lnot \to QPR$は論理式ではない。
  \end{itemize}
\end{ex}

\begin{dfn}
  論理式の記述法は他にもあり、ここではポーランド記法を紹介する。
  ここで、$n$は$\lnot$、$c$は$\to$、$k$は$\land$、$a$は$\lor$にそれぞれ対応している。
\end{dfn}

\begin{ex}(ポーランド記法)
  \begin{align}
    \lnot p &: np \\
    p \to q &: cpq \\
    p \land q &: kpq \\
    p \lor q &: apq \\
    (p \to q) \to ((q \to r) \to (p \to r)) &: ccpqccqrcpr
  \end{align}
\end{ex}

% --------------------------------------------形式言語の導入--------------------------------------------------





\pagebreak
% --------------------------------------------意味論---------------------------------------------------

\STEP{意味論}

ここでは論理式が、どの状況で真、偽になるかを定めていく。なお古典命題論理では
\begin{itemize}
  \item 排中律(Pである、またはPでないが成立する原理)
  \item 無矛盾律(正しくてなおかつ間違っていることはない)
\end{itemize}
の2つの哲学により、真、偽という2つの状態にのみ従うとしている。

\begin{dfn}
  言語$\mathcal{L}_{CPL}$に対する解釈(INTERPRETATION)とは、命題変数全体から\{T, F\}への関数である。
\end{dfn}

\begin{dfn}
  言語$\mathcal{L}_{CPL}$に対して解釈$I$が与えられたとき、論理式全体が\{T,F\}への(付値)関数$v$(valuation)を以下のように定める。
  \begin{equation}
    v(p) = I(p)
  \end{equation}
  ここでpは命題変数である。
\end{dfn}

\begin{ex}(付値)
  \begin{align}
    v(\lnot A) = T &\iff v(A) = F\\
    v(\lnot A) = F &\iff v(A) = T\\
    v(A \land B) = T &\iff v(A) = T \ かつ \ v(B) = T\\
    v(A \land B) = F &\iff v(A) = F \ または \ v(B) = F\\
    v(A \lor B) = F &\iff v(A) = T \ または \ v(B) = T\\
    v(A \lor B) = F &\iff v(A) = F \ かつ \ v(B) = F\\
    v(A \to B) = T &\iff v(A) = F \ または \ v(B) = T\\
    v(A \to B) = F &\iff v(A) = T \ かつ \ v(B) = F
  \end{align}
\end{ex}

% --------------------------------------------意味論---------------------------------------------------







% --------------------------------------------証明論---------------------------------------------------

\STEP{証明論}

証明論の展開方法は複数存在する。本講義では公理的方法を主に扱った。

\begin{dfn}
  古典命題論理の公理は以下の11個の図式からなる。
  \begin{flalign*}
    &\text{Ax.1}  \hspace{4cm} A \to (B \to A) & &\\
    &\text{Ax.2}  \hspace{4cm} (A \to (B \to C)) \to ((A \to B) \to (A \to C)) & &\\
    &\text{Ax.3}  \hspace{4cm} (A \land B) \to A & & \\
    &\text{Ax.4}  \hspace{4cm} (A \land B) \to B & & \\
    &\text{Ax.5}  \hspace{4cm} (C \to A) \to ((C \to B) \to (C \to (A \land B))) & & \\
    &\text{Ax.6}  \hspace{4cm} A \to (A \lor B) & & \\
    &\text{Ax.7}  \hspace{4cm} B \to (A \lor B) & & \\
    &\text{Ax.8}  \hspace{4cm} (A \to C) \to ((B \to C) \to ((A \lor B) \to C)) & & \\
    &\text{Ax.9}  \hspace{4cm} (A \to \lnot A) \to \lnot A & & \\
    &\text{Ax.10} \hspace{4cm} A \to (\lnot A \to B) & & \\
    &\text{Ax.11} \hspace{4cm} A \lor \lnot A & & \\
  \end{flalign*}

  これらに加えて、以下の推論規則を加える。
  MP(MODUS PONENS)
  \begin{equation*}
    \frac{A \  A \to B}{B}
  \end{equation*}
\end{dfn}

\begin{dfn}
  形式言語$\mathcal{L}_{CPL}$のもとで、体系HCPLに対して、$\vdash_{CPL}$を以下のように定める。\\
  $A1, \dots , An, B$を論理式とするとき、$A_1, \dots, A_n \vdash B \iff$ある論理式の列$C_1, \dots, C_n$があって、$Cn$は$B$で書き、$Ci$は以下の3つの条件のうちの1つを満たす。
  \begin{itemize}
    \item $C_i$は$A_1, \dots, A_n$のうちの1つ
    \item $C_i$はAx.1$\sim$Ax.11のうちの1つ
    \item $C_i$はすでに出現している2つの論理式に対して、MPを適用することで得られる。
  \end{itemize}
  このとき、$\vdash_{CPL}$を古典命題論理の証明論的帰結関係という。
\end{dfn}

% --------------------------------------------証明論---------------------------------------------------



% --------------------------------------------意味論と証明論の関係----------------------------------------------------

\STEP{意味論と証明論の関係}

授業を休んだため省略。


% --------------------------------------------意味論と証明論の関係----------------------------------------------------