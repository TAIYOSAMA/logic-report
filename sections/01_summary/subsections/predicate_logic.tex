\section{述語論理}

\subsection{古典述語論理}

\STEP{形式言語の導入}

\begin{dfn}
  古典述語論理の言語$\mathcal{L}_{PL}$(Predicate Logic)は次のものからなる。
  \begin{itemize}
    \item 個体変項: $x , y , z , \dots$
    \item 個体定項: $a, b, c, \dots$
    \item 述語: $P, Q, R, \dots$
    \item 論理結合子: $\lnot, \land, \lor, \to$
    \item 量化子: $\forall, \exists$
    \item カッコ: $(, )$
  \end{itemize}
\end{dfn}

\begin{dfn}
  言語$\mathcal{L}_{PL}$が与えられたとき、論理式を以下のように定める。
  \begin{itemize}
    \item $t_1, \dots, t_n$が項であり、$P$が$n$項連続のとき、$P_{t_1 \dots t_n}$は論理式である。(とくに原子式と呼ばれる)
    \item 2つの論理式$A, B$があるとき、$A$, $(A \land B)$, $(A \lor B)$, $(A \to B)$は論理式である。
    \item 論理式$A$と個体変項があるとき、$\forall_{x}A, \exists_{x}A$は論理式である。
  \end{itemize}
  上の3つの条件によってのみ論理式は決まる。
\end{dfn}

\STEP{意味論}

\begin{dfn}
  言語$\mathcal{L}_{PL}$が与えられているとき、個体変項$x$の出現が束縛されているとは、
  \begin{equation}
    \forall_x \dots x \dots
  \end{equation}
  あるいは、
  \begin{equation}
    \exists_x \dots x \dots
  \end{equation}
  のように出現することを言う。
  個体変項$x$の出現が束縛されていないとき、$x$は自由であるという。
  すべての個体変項の出現が束縛されている論理式を閉じた論理式という。
  また、論理式$A$において、自由に出現する個体変項$x$をすべて項$t$で置き換えて得られる論理式を$A_x(t)$とかく。
\end{dfn}

\begin{dfn}
  言語$\mathcal{L}_{PL}$に対する解釈とは、組$\mathfrak{I}=<D, I>$である。このとき、
  \begin{itemize}
    \item $D$: 空でない集合
    \item $I$: 関数であって、個体定項$C$に対して$I(C)$は$D$に含まれる元であり、$n$項述語$P$に対して、$I(P)$は$D^n$の部分集合
  \end{itemize}
  をそれぞれ対応させるものである。
\end{dfn}

\begin{dfn}
  言語$\mathcal{L}_{PL}$に対する$\mathfrak{I} = <D, I>$が与えられたとき、$\mathcal{L}_{PL}(\mathfrak{I})$における閉じた論理式全体から$\{T, F\}$への関数$v$を以下のように定める。
  \begin{itemize}
    \item $v({P_0}_{a_1 \dots a_n}) = T \iff <I(a_1), \dots, I(a_n)> \in I(P)$
    \item $v({P_0}_{a_1 \dots a_n}) = F \iff <I(a_1), \dots, I(a_n)> \notin I(P)$
  \end{itemize}
\end{dfn}

\begin{align}
  v(\lnot A) = T &\iff v(A) = F\\
  v(\lnot A) = F &\iff v(A) = T\\
  v(A \land B) = T &\iff v(A) = T \ かつ \ v(B) = T\\
  v(A \land B) = F &\iff v(A) = F \ または \ v(B) = F\\
  v(A \lor B) = F &\iff v(A) = T \ または \ v(B) = T\\
  v(A \lor B) = F &\iff v(A) = F \ かつ \ v(B) = F\\
  v(A \to B) = T &\iff v(A) = F \ または \ v(B) = T\\
  v(A \to B) = F &\iff v(A) = T \ かつ \ v(B) = F\\
  v(\forall_xA) = T &\iff すべてのDの元dに対して、v(A_x(k_d)) = T\\
  v(\forall_x A) = F &\iff ある D の元 d に対して、v(A_x(k_d)) = F\\
  v(\exists_x A) = T &\iff ある D の元 d に対して、v(A_x(k_d)) = T\\
  v(\exists_x A) = F &\iff すべての D の元 d に対して、v(A_x(k_d)) = F
\end{align}

\begin{dfn}
  $A_1, \dots, A_n, B$を閉じた論理式とする。このとき、
  $A_1, \dots, A_n \models_{PL} B \iff v(A_1) = T, \dots, v(A_n) = T かつ v(B) = F$となるような解釈$\mathfrak{I}$がない
  
  ここで、$\models_{PL}$を古典述語論理の意味論的帰結関係という。
\end{dfn}