\section{述語論理}

\subsection{古典述語論理}

\begin{dfn}
  古典述語論理の言語$\mathcal{L}_{PL}$(Predicate Logic)は次のものからなる。
  \begin{itemize}
    \item 個体変項: $x , y , z , \dots$
    \item 個体定項: $a, b, c, \dots$
    \item 述語: $P, Q, R, \dots$
    \item 論理結合子: $\lnot, \land, \lor, \to$
    \item 量化子: $\forall, \exists$
    \item カッコ: $(, )$
  \end{itemize}
\end{dfn}

\begin{dfn}
  言語$\mathcal{L}_{PL}$が与えられたとき、論理式を以下のように定める。
  \begin{itemize}
    \item $t_1, \dots, t_n$が項であり、$P$が$n$項連続のとき、$P_{t_1 \dots t_n}$は論理式である。(とくに原子式と呼ばれる)
    \item 2つの論理式$A, B$があるとき、$A$, $(A \land B)$, $(A \lor B)$, $(A \to B)$は論理式である。
    \item 論理式$A$と個体変項があるとき、$\forall_{x}A, \exists_{x}A$は論理式である。
  \end{itemize}
  上の3つの条件によってのみ論理式は決まる。
\end{dfn}