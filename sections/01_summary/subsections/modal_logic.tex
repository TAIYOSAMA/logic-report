\section{様相論理}

\subsection{様相論理とは}
様相論理とは、古典命題論理を拡張し、必然性や可能性といった様相概念を扱う論理である。
古典命題論理では命題の真偽を1つの解釈の下で考えるのに対し、
様相論理ではこれらの様相概念を解釈するために、複数の可能世界を用いた意味論が導入される。

\subsection{様相論理}
様相論理(Modal Logic)についてまとめる。

\STEP{形式言語の導入}

\begin{dfn}
  様相論理の言語$\mathcal{L}_{ML}$(Modal Logic)は以下のものからなる。
  \begin{itemize}
    \item 命題変数: $p, q, r, \dots$
    \item 論理結合子: $\lnot, \land, \lor, \to $
    \item カッコ: $\lparen, \rparen$
    \item 様相演算子: $\Box, \Diamond$
  \end{itemize}
  ここで様相演算子はそれぞれ、$\Box$は必然性を、$\Diamond$は可能性を表す。
\end{dfn}

\begin{dfn}
  様相論理にもポーランド記法が存在し、以下のように記述する。
  ここで、$\Box$はlに、$\Diamond$はmにそれぞれ対応している。
\end{dfn}

\begin{ex}
  \begin{align}
    \Box p &: lp \\
    \Diamond \lnot p &: lnp \\
    \Box (p \to q) &: lcpq \\
    \Box p \to \Diamond q &: clpmq
  \end{align}
\end{ex}

