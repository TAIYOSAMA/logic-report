\section{様相論理}

\subsection{様相論理とは}
様相論理とは、古典命題論理を拡張し、必然性や可能性といった様相概念を扱う論理である。
古典命題論理では命題の真偽を1つの解釈の下で考えるのに対し、
様相論理ではこれらの様相概念を解釈するために、複数の可能世界を用いた意味論が導入される。

\subsection{様相論理}
様相論理(Modal Logic)についてまとめる。

\STEP{形式言語の導入}

\begin{dfn}
  様相論理の言語$\mathcal{L}_{ML}$(Modal Logic)は以下のものからなる。
  \begin{itemize}
    \item 命題変数: $p, q, r, \dots$
    \item 論理結合子: $\lnot, \land, \lor, \to $
    \item カッコ: $\lparen, \rparen$
    \item 様相演算子: $\Box, \Diamond$
  \end{itemize}
  ここで様相演算子はそれぞれ、$\Box$は必然性を、$\Diamond$は可能性を表す。
\end{dfn}

\begin{dfn}
  様相論理にもポーランド記法が存在し、以下のように記述する。
  ここで、$\Box$はlに、$\Diamond$はmにそれぞれ対応している。
\end{dfn}

\begin{ex}
  \begin{align}
    \Box p &: lp \\
    \Diamond \lnot p &: lnp \\
    \Box (p \to q) &: lcpq \\
    \Box p \to \Diamond q &: clpmq
  \end{align}
\end{ex}

\STEP{意味論}

ここでは可能世界意味論もしくはクリプト意味論とも呼ばれるものを紹介する。
これはあくまで代表的なものであり、唯一絶対の意味論ではないことに注意したい。

\begin{dfn}
  言語$\mathcal{L}_{ML}$に対する$S5$モデルとは、組<$W$, $@$, $I$>である。このとき、
  \begin{itemize}
    \item $W$は空でない集合
    \item $@$はWに含まれる元のうちの1つ
    \item $I$は$W$と命題変数から\{$T$, $F$\}への関数
  \end{itemize}
\end{dfn}

\begin{rem}
  ルイスとラングフォードは$S1$、$S2$、$S3$、$S4$、$S5$の5つの体系を証明論の立場から導入した。ここでは特に$S5$を扱う。
\end{rem}

\begin{dfn}
  言語$\mathcal{L}_{ML}$に対して、$S5$モデル<$W$, $@$, $I$>が与えられたとき、$W$と論理式全体から\{$T$, $F$\}への関数$v$を以下のように定める。
  すべての$W$の元($w$)に対して、
  \begin{equation}
    v(w, p) = I(w, p)(pは命題変数)
  \end{equation}
  また、
  \begin{align}
    v(w, \lnot A) = T &\iff v(w, A) = F \\
    v(w, \lnot A) = F &\iff v(w, A) = T \\[6pt]
    v(w, A \land B) = T &\iff v(w, A) = T \ \text{かつ} \ v(w, B) = T \\
    v(w, A \land B) = F &\iff v(w, A) = F \ \text{または} \ v(w, B) = F \\[6pt]
    v(w, A \lor B) = T &\iff v(w, A) = T \ \text{または} \ v(w, B) = T \\
    v(w, A \lor B) = F &\iff v(w, A) = F \ \text{かつ} \ v(w, B) = F \\[6pt]
    v(w, A \to B) = T &\iff v(w, A) = F \ \text{または} \ v(w, B) = T \\
    v(w, A \to B) = F &\iff v(w, A) = T \ \text{かつ} \ v(w, B) = F \\[6pt]
    v(w, \Box A) = T &\iff \text{すべての } W \text{ の元 } x \text{ に対して } v(x, A) = T \\
    v(w, \Box A) = F &\iff \text{ある } W \text{ の元 } x \text{ に対して } v(x, A) = F \\[6pt]
    v(w, \Diamond A) = T &\iff \text{ある } W \text{ の元 } x \text{ に対して } v(x, A) = T \\
    v(w, \Diamond A) = F &\iff \text{すべての } W \text{ の元 } x \text{ に対して } v(x, A) = F
  \end{align}
\end{dfn}

\STEP{証明論}

\begin{dfn}
  形式言語$\mathcal{L}_{ML}$のもとで体系$HS5$を以下のように定める。
  \begin{flalign*}
    &\text{Ax.1}  \hspace{4cm} A \to (B \to A) & &\\
    &\text{Ax.2}  \hspace{4cm} (A \to (B \to C)) \to ((A \to B) \to (A \to C)) & &\\
    &\text{Ax.3}  \hspace{4cm} (A \land B) \to A & & \\
    &\text{Ax.4}  \hspace{4cm} (A \land B) \to B & & \\
    &\text{Ax.5}  \hspace{4cm} (C \to A) \to ((C \to B) \to (C \to (A \land B))) & & \\
    &\text{Ax.6}  \hspace{4cm} A \to (A \lor B) & & \\
    &\text{Ax.7}  \hspace{4cm} B \to (A \lor B) & & \\
    &\text{Ax.8}  \hspace{4cm} (A \to C) \to ((B \to C) \to ((A \lor B) \to C)) & & \\
    &\text{Ax.9}  \hspace{4cm} (A \to \lnot A) \to \lnot A & & \\
    &\text{Ax.10} \hspace{4cm} A \to (\lnot A \to B) & & \\
    &\text{Ax.11} \hspace{4cm} A \lor \lnot A & & \\
    &\text{Ax.12} \hspace{4cm} \Box (A \to B) \to (\Box A \to \Box) & & \\
    &\text{Ax.13} \hspace{4cm} \Box A \to A & & \\
    &\text{Ax.14} \hspace{4cm} \Box A \to \Box \Box A & & \\
    &\text{Ax.15} \hspace{4cm} \lnot \Box A \to \Box \lnot \Box A & & \\
    &\text{Ax.16} \hspace{4cm} \Diamond A \to \lnot \Box \lnot A & & \\
    &\text{Ax.17} \hspace{4cm} \lnot \Box \lnot A \to \Diamond A & & \\
  \end{flalign*}
  以上の17個の公理図式に加えて、以下の2つの推論規則を加える。
  \begin{flalign*}
    &\text{MP} \hspace{4cm} \frac{A \ A \to B}{B} & & \\
    &\text{RN} \hspace{4.5cm} \frac{A}{\Box A} & & \\
  \end{flalign*}
  ここで、RN(Rule of Necessation)は必然化則である。
\end{dfn}